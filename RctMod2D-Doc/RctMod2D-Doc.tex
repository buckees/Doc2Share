%% LyX 2.3.0 created this file.  For more info, see http://www.lyx.org/.
%% Do not edit unless you really know what you are doing.
\documentclass[english]{article}
\usepackage[T1]{fontenc}
\usepackage[latin9]{inputenc}
\usepackage{amsmath}
\usepackage{setspace}
\setstretch{1.5}

\makeatletter
%%%%%%%%%%%%%%%%%%%%%%%%%%%%%% Textclass specific LaTeX commands.
\newcommand{\lyxaddress}[1]{
	\par {\raggedright #1
	\vspace{1.4em}
	\noindent\par}
}

\makeatother

\usepackage{babel}
\begin{document}

\title{\noindent The Langmuir Reactor Model}

\author{\noindent Wei Tian}
\maketitle

\lyxaddress{\noindent TSMC}
\begin{abstract}
\noindent In this document, a reactor scale model is developed and
used for recipe tuning. The detailed physical mechanisms and numerical
algorithms are introduced and discussed.
\end{abstract}

\section{Introduction}

The modeling of plamsas for semiconductor processes is challegened
by conflicting goals - having detailed, specialized algorithms resolving
the particular problems such as iso-dense problems and metal contaminations
while also being fast enough to scan a wide parameter space for recipe
tuning. The former challenge requires the model as accuracy as possible
in physics. However, the latter challenge requires the model as fast
as possible in computing. Scrifice and comprise have to be made between
these two. The Langmuir Reactor Model is developed to meet these challenges. 

\section{Geometry and Mesh}

\begin{onehalfspace}
\noindent The mesh in Langmuir Reactor Model is constructed in 2D
space, in which $(x,z)$ are used to represent the 2D coordinates.
( to be edited: and infinity is assumed in $y$ direction. The model
discretizes the 2D space into a rectangular computational cells. The
cell center is marked as a node, which determines the location $(x,z)$
of the cell. Each cell has a volume of $\Delta x\times\Delta z$,
where $\Delta x$ and $\Delta z$ are the resolutions in $x$ and
$z$ directions, respectively. Usually, square cells, where $\Delta x=\Delta z$,
are used. Non-square cells, which are used for high aspect ratio domain
for memory saving, need future test and validation. The computational
complexity increases with reducing resolution as approximately $O(n^{3})$,
where $n$ is the number of cells per side in the simulation domain.)
\end{onehalfspace}

\section{Poisson Equation}

\subsection{Explicit Poisson's equation}

With explict method, the potenital in the Poisson's equation is determined
by the current charge density, which is impacted by the prevous potential.
\[
-\nabla\cdot\varepsilon\nabla\phi(t)=e(\sum_{ion}n_{i}(t)-n_{e}(t))
\]
\[
-\nabla\cdot\varepsilon\nabla\phi(t+\Delta t)=e(\sum_{ion}n_{i}(t+\Delta t)-n_{e}(t+\Delta t))
\]
\[
\dfrac{\partial n_{e,i}}{\partial t}=D_{e,i}\nabla^{2}n_{e,i}(t)\pm\nabla\cdot(\mu_{e,i}n_{e,i}(t)\nabla\phi(t))+S_{e}(t)
\]

\[
n_{e,i}(t+\Delta t)=n_{e,i}(t)+\Delta t\times f_{e,i}(\phi(t))
\]
\[
-\nabla\cdot\varepsilon\nabla\phi(t+\Delta t)=F(\phi(t))
\]

where
\[
\phi=potenital
\]
\[
\varepsilon=permittivity
\]
\[
e=elementary\:charge
\]

\[
n_{e,i}=electron,ion\:density
\]
\[
\mu_{e,i}=electron,ion\:mobility
\]

with the derivation above, the future potential is determined by the
current potenital.

\subsection{Semi-implicit electron with predictor-corrector ions}

The timestep is limited to as small as a few picoseconds, which is
not practical for any useful simulations. Implicit method can theoretically
remove the limit of the timestep, with the cost of solving for the
reversed matrix. The matrix solver could be much expensive as well.
A semi-implict method is, therefore, proposed to increase the timestep
and meanwhile reduce the cost of matrix solver. The principal is shown
as below, 
\[
-\nabla\cdot\varepsilon\nabla\phi(t+\Delta t)=e(\sum_{ion}n_{i}(t+\Delta t)-n_{e}(t+\Delta t))
\]
\[
\dfrac{\partial n_{e}}{\partial t}=D_{e}\nabla^{2}n_{e}(t)-\nabla\cdot(\mu_{e}n_{e}(t)\nabla\phi(t+\Delta t))+S_{e}(t)
\]
\[
n_{e}(t+\Delta t)=n_{e}(t)+\Delta t\times f_{e}(\phi(t+\Delta t))
\]
\[
\dfrac{\partial n_{i}}{\partial t}=D_{i}\nabla^{2}n_{i}(t)+\nabla\cdot(\mu_{i}n_{i}(t)\nabla\phi(t))
\]
\[
n_{i}(t+\Delta t)=n_{i}(t)+\Delta t\times f_{i}(\phi(t))
\]
\[
-\nabla\cdot\varepsilon\nabla\phi(t+\Delta t)=e(\sum_{ion}[n_{i}(t)+f_{i}(\phi(t))]-[n_{e}(t)+f_{e}(\phi(t+\Delta t))])
\]
\[
-\nabla\cdot\varepsilon\nabla\phi(t+\Delta t)=
\]
\[
e(\sum_{ion}[\underline{n_{i}(t)+\Delta t\times f_{i}(\phi(t))}]-\Big[\underline{n_{e}(t)+\Delta t\times S_{e}(t)+\Delta t\times D_{e}\nabla^{2}n_{e}(t)-\Delta t\times\nabla\cdot(\mu_{e}n_{e}(t)\nabla\phi(t+\Delta t))}\Big])
\]
\[
-\nabla\cdot\varepsilon\nabla\phi(t+\Delta t)+e\times\Delta t\times\nabla\cdot(\mu_{e}n_{e}(t)\nabla\phi(t+\Delta t))=F(n_{e,i}(t),\phi(t))
\]
\[
\Big[\underline{-\nabla\varepsilon\cdot\nabla\phi(t+\Delta t)-\varepsilon\nabla^{2}\phi(t+\Delta t)}\Big]+\Big[\underline{(e\mu_{e}\Delta t)\nabla n_{e}(t)\cdot\nabla\phi(t+\Delta t)+(e\mu_{e}\Delta t)n_{e}(t)\nabla^{2}\phi(t+\Delta t)}\Big]
\]
\[
=F(n_{e,i}(t),\phi(t))
\]
\[
[\underline{(e\mu_{e}\Delta t)n_{e}(t)-\varepsilon}]\nabla^{2}\phi(t+\Delta t)+[\underline{(e\mu_{e}\Delta t)\nabla n_{e}(t)-\nabla\varepsilon}]\cdot\nabla\phi(t+\Delta t)=F(n_{e,i}(t),\phi(t))
\]
\[
\underline{A(n_{e}(t))}\nabla^{2}\phi(t+\Delta t)+\underline{\nabla B(n_{e}(t))}\cdot\nabla\phi(t+\Delta t)=F(n_{e,i}(t),\phi(t))
\]
\[
underline-emphasis
\]

The electron density is solved using implicit method, where the future
electron density is determined by the future potential, while the
ion density is still solved using explicit method, where the future
ion density is determined by the current potential. In this scenario,
electron denisty and potential get quickly convergent with large timestep.
A further attension needs to be payed to the ion density, which could
oscillate. The predictor-corrector method is used for ion density.
\[
\tilde{n}_{i}(t+\Delta t)=n_{i}(t)+\Delta t\times f_{i}(n_{i}(t),\phi(t))
\]
\[
n_{i}(t+\Delta t)=n_{i}(t)+\Delta t\times\dfrac{1}{2}(f_{i}(n_{i}(t),\phi(t))+f_{i}(\tilde{n}_{i}(t+\Delta t),\phi(t+\Delta t)))
\]


\subsection{Semi-implicit Poisson's equation}

The ion density can use implicit method as well. In this case, 
\[
-\nabla\cdot\varepsilon\nabla\phi(t+\Delta t)=e(\sum_{ion}[\underline{n_{i}(t)+f_{i}(\phi(t+\Delta t))}]-[\underline{n_{e}(t)+f_{e}(\phi(t+\Delta t))}])
\]
\[
-\nabla\cdot\varepsilon\nabla\phi(t+\Delta t)=e(\sum_{e,i}[\underline{n_{e,i}(t)+f_{e,i}(\phi(t+\Delta t))}]
\]
\[
-\nabla\cdot\varepsilon\nabla\phi(t+\Delta t)=
\]
\[
e(\sum_{e,i}\Big[\underline{n_{e,i}(t)+\Delta t\times S_{e,i}(t)+\Delta t\times D_{e,i}\nabla^{2}n_{e,i}(t)+\Delta t\times\nabla\cdot(q\mu_{e,i}n_{e,i}(t)\nabla\phi(t+\Delta t))}\Big])
\]
\[
-\nabla\cdot\varepsilon\nabla\phi(t+\Delta t)-\sum_{e,i}\Delta t\times\nabla\cdot(eq\mu_{e,i}n_{e,i}(t)\nabla\phi(t+\Delta t))=F(n_{e,i}(t))
\]
\[
[\underline{\sum_{e,i}(eq\mu_{e,i}\Delta t)n_{e,i}(t)-\varepsilon}]\nabla^{2}\phi(t+\Delta t)+[\underline{(e\Delta t)\sum_{e,i}\mu_{e,i}\nabla n_{e,i}(t)-\nabla\varepsilon}]\cdot\nabla\phi(t+\Delta t)=F(n_{e,i}(t))
\]
\[
\underline{A(n_{e,i}(t))}\nabla^{2}\phi(t+\Delta t)+\underline{\nabla B(n_{e,i}(t))}\cdot\nabla\phi(t+\Delta t)=F(n_{e,i}(t))
\]

The future potential is solved directly, which means that the future
potential does not depends on the current potential anymore. All coefficients
in this equation are densities at current time. The second term in
the LHS is probably (not sure, need addtional check) not easy to deal
with. It could be rewritten in explicit form and moved to the RHS.
\[
\underline{A(n_{e,i}(t))}\nabla^{2}\phi(t+\Delta t)+\underline{\nabla B(n_{e,i}(t))}\cdot\nabla\phi(t+\Delta t)=F(n_{e,i}(t))
\]
\[
\underline{A(n_{e,i}(t))}\nabla^{2}\phi(t+\Delta t)=F(n_{e,i}(t))+\underline{\nabla B(n_{e,i}(t))}\cdot\nabla\phi(t)
\]
\[
\underline{A(n_{e,i}(t))}\nabla^{2}\phi(t+\Delta t)=F(n_{e,i}(t),\phi(t))
\]

Now the Poisson's equation retains its Lapalacian form, with the source
term depending on current potential. The same technique can be applied
to Sec. 3.2 as well.
\end{document}
